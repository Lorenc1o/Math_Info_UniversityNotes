%% LyX 2.2.3 created this file.  For more info, see http://www.lyx.org/.
%% Do not edit unless you really know what you are doing.
\documentclass[11pt,english]{article}
\usepackage[T1]{fontenc}
\usepackage[latin9]{inputenc}
\usepackage[a4paper]{geometry}
\geometry{verbose,tmargin=3cm,bmargin=3cm,lmargin=2cm,rmargin=2cm,headheight=3cm,headsep=3cm,footskip=2cm}
\usepackage{amsmath}
\usepackage{amssymb}

\makeatletter
\@ifundefined{date}{}{\date{}}
\makeatother

\usepackage{babel}
\begin{document}

\title{GyA - Tarea 2}

\author{Jose Antonio Lorencio Abril}

\date{3/04/2020}
\maketitle

\subsubsection*{3.2.16. En el problema 2.1.12 se ha visto que el cardinal de un cuerpo
finito K es una potencia de un n�mero primo (de hecho una potencia
de la caracter�stica de K). En este problema, fijado un entero primo
positivo $\boldsymbol{p}$, vamos a ver que existen cuerpos de cardinal
$\boldsymbol{p^{n},\forall n\in\mathbb{Z}^{+}}$.}
\begin{flushleft}
\textbf{(1) Sea $\boldsymbol{K}$ un cuerpo de caracter�stica $\boldsymbol{p,n\in\mathbb{Z}^{+}}$.
Demostrar que el conjunto de las ra�ces en $\boldsymbol{K}$ del polinomio
$\boldsymbol{x^{p^{n}}-x}$ es un subcuerpo finito de $\boldsymbol{K}$.}
\par\end{flushleft}

\begin{flushleft}
El ejercicio 2.1.11 nos dice que 
\[
\begin{array}{c}
f:\ K\rightarrow K\\
\quad x\mapsto x^{p^{n}}
\end{array}
\]
\par\end{flushleft}

\begin{flushleft}
es un endomorfismo de $K,\forall n\in\mathbb{Z}^{+}$.
\par\end{flushleft}

\begin{flushleft}
Sea $A=\{a\in K:\ a^{p^{n}}-a=0\}$
\par\end{flushleft}

\begin{flushleft}
�Es $A$ un subcuerpo de $K$? Lo ser� si, y solo si, es un subanillo.
Ve�moslo.
\par\end{flushleft}
\begin{itemize}
\item \begin{flushleft}
�$1\in A$?
\par\end{flushleft}

\begin{flushleft}
\[
1^{p^{n}}-1=1-1=0\qquad\surd
\]
\par\end{flushleft}
\item \begin{flushleft}
�$a,b\in A\implies a+b\in a$?
\par\end{flushleft}

\begin{flushleft}
\[
(a+b)^{p^{n}}-(a+b)=\sum_{k=0}^{p^{n}}\left(\begin{array}{c}
p^{n}\\
k
\end{array}\right)a^{p^{n}-k}b^{k}-(a+b)=
\]
\[
=a^{p^{n}}+\left(\begin{array}{c}
p^{n}\\
1
\end{array}\right)a^{p^{n}-1}b+...+\left(\begin{array}{c}
p^{n}\\
p^{n}-1
\end{array}\right)ab^{p^{n}-1}+b^{p^{n}}-(a+b=*
\]
 Todos los factores entre $a^{p^{n}},b^{p^{n}}$ tienen al menos una
$p$ multiplicando, porque $p$ es primo y, por tanto, $p^{n}$ siempre
dejar� un factor $p$, pues su mayor divisor distinto de s� mismo
es $p^{n-1}$. Por tanto
\[
*=a^{p^{n}}+b^{p^{n}}-(a+b)=(a^{p^{n}}-a)+(b^{p^{n}}-b)=0-0=0\qquad\surd
\]
\par\end{flushleft}
\item \begin{flushleft}
�$a,b\in A\implies a\cdot b\in A$?
\par\end{flushleft}

\begin{flushleft}
Primero n�tese que
\[
a^{p^{n}}-a=0\iff a^{p^{n}}=a
\]
\[
b^{p^{n}}-b=0\iff b^{p^{n}}=b
\]
\par\end{flushleft}

\begin{flushleft}
Entonces
\[
(ab)^{p^{n}}-ab=a^{p^{n}}b^{p^{n}}-ab=ab-ab=0\qquad\surd
\]
\par\end{flushleft}

\end{itemize}
\begin{flushleft}
As�, $A$ es subanillo de $K$. Por lo que $A$ es subcuerpo de $K$.
Como $K$ es finito, entonces $A$ tambi�n lo es.
\par\end{flushleft}

\begin{flushleft}
\textbf{(2) Deducir que, $\boldsymbol{\forall n\in\mathbb{Z}^{+}}$,
existe un cuerpo de cardinal $\boldsymbol{p^{n}}$.}
\par\end{flushleft}

\begin{flushleft}
Sea $K$ un cuerpo de caracter�stica $p$.
\[
x^{p^{n}}-x\in K[x]-K
\]
\par\end{flushleft}

\begin{flushleft}
As�, por el ejercicio 3.2.15, existe un cuerpo $K'$ que contiene
a $K$ como subcuerpo y $P$ es producto de polinomios de grado 1
con coeficientes en $K$.
\par\end{flushleft}

\begin{flushleft}
Como el grado del polinomio es $p^{n}$, entonces ser� producto de
$p^{n}$ polinomios de grado 1. Es decir, tendr� $p^{n}$ ra�ces.
\par\end{flushleft}

\begin{flushleft}
Por el apartado 1, el conjunto $A$ de estas ra�ces es un cuerpo.
Al haber $p^{n}$ ra�ces, $|A|=p^{n}$.
\par\end{flushleft}

\subsubsection*{3.3.1. �Es cierto que, si $\boldsymbol{D}$ es un DFU y $\boldsymbol{b}$
es un elemento de $\boldsymbol{D}$, entonces solo hay una cantidad
finita de ideales de $\boldsymbol{D}$ que contienen a $\boldsymbol{b}$?
�Y si $\boldsymbol{D}$ es DIP?}
\begin{flushleft}
Veamos primero el caso en que \textbf{$\boldsymbol{D}$ es DIP}, en
particular, tambi�n es DFU.
\par\end{flushleft}

\begin{flushleft}
Entonces 
\[
b=u\cdot p_{1}\cdot...\cdot p_{n}
\]
\par\end{flushleft}

\begin{flushleft}
De forma �nica. Esto quiere decir que
\[
b\in(u)=D,(p_{1}),...,(p_{n})
\]
\par\end{flushleft}

\begin{flushleft}
y a sus intersecciones. Como $D$ es DIP, estos son todos los ideales
que lo contienen. En efecto, supongamos $I\vartriangleleft D$ distinto
de los anteriores.
\par\end{flushleft}

\begin{flushleft}
Entonces, como es DIP, $I=(a)$.
\[
b\in(a)\iff up_{1}\cdot...\cdot p_{n}=b=c\cdot a\overset{D\ DFU}{\implies}\begin{cases}
c=u_{c}p_{c1}\cdot...\cdot p_{cm}\\
a=v_{a}p_{a1}\cdot...\cdot p_{ak}\\
m+k=n
\end{cases}
\]
\par\end{flushleft}

\begin{flushleft}
En concreto, se tiene que $(a)=(p_{a1}\cdot...\cdot p_{ak})=(p_{a1})\cap...\cap(p_{ak})$.
Por lo que $I$ es como los anteriores.
\par\end{flushleft}

\begin{flushleft}
Si $\boldsymbol{D}$ \textbf{es DFU}. Vamos a ver un contraejemplo.
\[
\mathbb{Z}\ DFU\iff\mathbb{Z}[x]\ DFU\iff\mathbb{Z}[x][y]\ DFU
\]
\par\end{flushleft}

\begin{flushleft}
Pero este �ltimo no es DIP, por la proposici�n 3.13.
\par\end{flushleft}

\begin{flushleft}
Es m�s,
\[
x\in(x,y),(x,y^{2}),(x,y^{3}),...
\]
\par\end{flushleft}

\begin{flushleft}
Una cantidad infinita de ideales, distintos, pues $y^{n-1}\notin(x,y^{n})$.
\par\end{flushleft}

\subsubsection*{3.4.1. Sea $\boldsymbol{D}$ un DFU y sea $\boldsymbol{f=a_{0}+a_{1}x+...+a_{n}x^{n}}$
un polinomio primitivo en $\boldsymbol{D[x]}$. Demostrar que, si
existe un irreducible $\boldsymbol{p\in D}$ tal que
\[
\boldsymbol{p|a_{i}\ \forall i>0,\qquad p\nmid a_{0},\qquad p^{2}\nmid a_{n}}
\]
 entonces $\boldsymbol{f}$ es irreducible en $\boldsymbol{D[x]}$.}
\begin{flushleft}
Pensemos $f=g\cdot h$. �Ser� $gr(g)=n$ � $gr(f)=n$?
\[
g=b_{0}+...+b_{m}x^{m},\qquad h=c_{0}+...+c_{k}x^{k},\qquad b_{m}c_{k}\neq0
\]
\par\end{flushleft}

\begin{flushleft}
Por otro lado, 
\[
p^{2}\nmid a_{n}=b_{m}c_{k}\implies p\nmid b_{m}\qquad\acute{o}\qquad p\nmid c_{k}
\]
\par\end{flushleft}

\begin{flushleft}
Supongamos que $p\nmid c_{k}$. Como $f$ es primitivo, entonces $p\nmid g$,
pues si $p|g\overset{g|f}{\implies}p|f\implies f\ no\ primitivo$,
pero esto no es as�.
\par\end{flushleft}

\begin{flushleft}
Entonces, tomamos
\[
i=\max\{j:p\nmid b_{j}\}
\]
\par\end{flushleft}

\begin{flushleft}
Consideremos ahora
\[
a_{i+k}=\sum_{j=0}^{i+k-1}b_{j}c_{i+k-j}+b_{i+k}c_{0}
\]
\par\end{flushleft}

\begin{flushleft}
Como $gr(c)=k$, $c_{i+k-j}=0\:\forall j<i$. Por lo que 
\[
a_{i+k}=\sum_{j=i}^{i+k-1}b_{j}c_{i+k-j}+b_{i+k}c_{0}
\]
\par\end{flushleft}

\begin{flushleft}
Entonces, tenemos que $p|b_{j}c_{i+k-j},\ \forall j>i$, ya que $i$
es el m�ximo de los \textbf{$b_{j}$ }que no son divisibles por $p$.
Es decir, $p$ divide a todos los sumandos de $a_{i+k}$, excepto
a $b_{i}c_{k}$:
\[
a_{i+k}=b_{i}c_{k}+S\cdot p
\]
\par\end{flushleft}

\begin{flushleft}
Donde $S$ es lo que queda al sacar $p$ factor com�n en todo el sumatorio.
Tenemos, de esta manera, que $p\nmid a_{i+k}$, pero esto quiere decir
que $i+k=0$. Al ser ambos n�meros no negativos, queda $i=0=k$. Es
decir, $gr(h)=0\implies gr(g)=n$
\par\end{flushleft}

\begin{flushleft}
Si hacemos el otro caso, $p\nmid b_{m}$, obtendremos $gr(h)=n$.
\par\end{flushleft}

\begin{flushleft}
Tal y como quer�amos ver, $f$ es irreducible.
\par\end{flushleft}
\end{document}
